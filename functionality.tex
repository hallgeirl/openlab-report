\section{Project Overview and Functionality}

The project consists of two main parts:
\begin{itemize}
\item checksumd is the checksum daemon, written in Python. 
\item checksum-monitor is a monitor written in Django that parses log files from checksumd to generate statistics and display details about checksum errors.
\end{itemize}

In addition there is the Lemon integration, but this was purely a configuration task. Please see section \ref{sec:lemon_integration} for details. 

\subsection{Overview of architecture}
An overview of the project can be seen in figure \ref{fig:complete_overview}. The checksumd program runs as a separate application on each disk server. For each file, it computes a checksum, and if there is an error it outputs this to an error log that resides in a folder in AFS (Andrew File System, a distributed file system used at CERN\cite{afs}). These entries consist of the filename, file size, the checksums, the path returned from the name server, the specifications of the machine (hardware model, hostname, etc.) and more. The same entry is also put in the syslog. 

The checksum-monitor Django application parses the error logs that were stored on AFS. This can be set up as a cron job, or done manually through the web interface. The application will then parse the logs, organize the entries in a relational database, and display it to the clients according to their query preferences.

With regular intervals (e.g. every 24 hours) a summary is also written to the syslog. This summary consists of, among other things, the aggregated numbers for the number of files checked, bad checksums, good checksums, etc. This is then parsed by the Lemon sensor, which is passed to the agent which passes it to the central Lemon database.

\begin{figure}[ht]
\centering
\includegraphics[width=0.7\textwidth]{gfx/complete_overview}
\caption{Architecture overview}
\label{fig:complete_overview}
\end{figure}

\subsection{checksumd}
\label{sec:checksumd_overview}
The checksumd daemon is implemented as a standalone application in Python. In principle, it can be used for other types of machines than CASTOR disk servers, but that is not within the scope of this project. This section gives an overview of the daemon.

\subsubsection{Daemon overview}

The process by which the daemon checks files is first by building a queue of all the disks. Each disk has a list of files to be checked (all files under the specified directory, that are not recently modified), and each has an index that points to the next file on the disk. Each of the mount points is assumed to be on a separate disk. At regular intervals, the main thread refreshes the file lists by querying for all files again, adding new ones and removing deleted ones. 

checksumd is built with threading in mind. There are two types of threads: the main thread, and worker threads. The worker threads do the actual integrity checks, while the main thread is responsible for housekeeping (e.g. keeping the file list updated and writing summaries to the log). The number of worker threads can be specified on the command line.

Until every file on each disk has been checked, each thread will grab one disk, check if it is busy, and if it is not, compute the checksum of the files. If the disk becomes busy while checking, the thread working on that disk will complete the current file it is working on and move on to the next disk. If no more disks are remaining in the queue, the threads will sleep for a specified amount of time before attempting to check again, continuing where they left off. The process is modelled in figure \ref{fig:checksum_process}.

\begin{figure}[ht]
\centering
\includegraphics[width=0.5\textwidth]{gfx/disk_check_process}
\caption{Process for selection of files for checking}
\label{fig:checksum_process}
\end{figure}

\subsection{checksum-monitor}
A monitor for individual events (errors) and statistics over multiple events has been developed. In addition to displaying the checksum errors from checksumd, it may also display a list of resolved events (e.g. events that the user has been notified about but that were impossible to fix), as well as "lost" files, files that were lost on disk servers due to crashes or other events.  The monitor also supports, in addition to listing every event, searching over a specified time frame, on host name, hardware model, cluster name, and so on. It also displays statistics like counting events per host name, hardware model, and user group. The monitor was fully developed in Python using Django. 

A screenshot from the web interface is shown in figure \ref{fig:monitor-index}. Statistics are shown to the right, grouped on hostname, hardware model and user group. Events can be filtered based on age (e.g. less than 24 hours, less than a week, etc) on hostname (by clicking each hostname), on hardware model, or on group. Since there may be a lot of errors (in e.g. the Lost files category), paging was also implemented, where only a subset of the events are displayed at once. Up to 1000 events can be shown on one page. To switch categories, one can click "From checksumd" for checksumd errors, "Fixed" for resolved errors and "Lost files" for files from the lost files logs. 

The entries can be sorted by the fields with a blue color in the header. Clicking the header will sort by that field, and clicking it again will reverse the sorting direction. 

\begin{figure}[ht]
\centering
\includegraphics[width=\textwidth]{gfx/checksum-monitor-main}
\caption{Checksum monitor main page}
\label{fig:monitor-index}
\end{figure}

In addition to the main page, there is a search page which can be reached either by clicking "Query" in the menu or "Custom query" near the timespan buttons. The search form is shown in figure \ref{fig:monitor-search}. All the fields that are searched on are AND-ed, forming the final query. So specifying e.g. hostname "lxfsj4107" and category "fixed" searches for all events from host lxfsj4107 that are in the fixed category.

\begin{figure}[ht]
\centering
\includegraphics[width=0.5\textwidth]{gfx/checksum-monitor-search}
\caption{Checksum monitor search page}
\label{fig:monitor-search}
\end{figure}

