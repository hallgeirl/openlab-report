\section{Configuration}

\subsection{checksumd}
All of checksumd's configuration is done by command line arguments. checksumd can be run in two modes: as a daemon, or as a regular command line tool. The available command line arguments are listed in table \ref{tab:cmdlineargs}.


A default configuration file, /etc/sysconfig/checksumd is shipped with the software package containing the daemon. This is a basic shell script that puts the command line arguments into an environment variable \$CHECKSUMD\_OPTIONS that is used by checksumd's runscript (typically /etc/init.d/checksumd). E.g., setting \$CHECKSUMD\_OPTIONS to -{}-maxwait 90, causes the runscript to run "checksumd -{}-maxwait 90 -{}-daemon" (daemon mode is implied by the init.d script).

\begin{table}[H]
\begin{tabular}{lp{10cm}}
\hline
{\bf Argument}              & {\bf Description}\\
\hline
{\tt -{}-daemon}              & Run checksumd as a daemon.\\
{\tt -{}-loopcount N}         & Run for N loops. Default: -1 (infinite) in daemon mode and 1 otherwise.\\
{\tt -{}-nodlf}               & Disable output to syslog.\\
{\tt -{}-checkns}             & Also check the content checksum against the one stored in the name server.\\
{\tt -{}-throttlesleep N}     & Sleep for N seconds if all disks are found to be busy (or done), before doing another busy check. Default: 10 sec.\\
{\tt -{}-maxruntime N}        & Terminate the program after N seconds. Default: infinite\\
{\tt -{}-maxwait N}           & If disks are busy and have M open files, force a check after $N\cdot 2^{M-1}$ seconds. Default: -1 (infinity)\\
{\tt -{}-minrepeattime N}     & If checking all files takes T < N seconds, sleep for N-T seconds before checking them again. Default: 24 hours\\
{\tt -{}-setchecksumifnone}   & Enable in order to set the checksum of files that do not have one.\\
{\tt -{}-modifyignore N}      & Ignore files that have been modified less than N seconds ago. Default: 24 hours\\
{\tt -{}-errorfile DIR}       & Save error logs under the directory DIR. Filenames are given as {\tt checksum\_errors\_<hostname>.log}.\\
{\tt -{}-nthreads N}          & Use N worker threads for integrity checking. This does not include the main thread. Default: 2\\
{\tt -{}-refreshrate N}       & Refresh the file queues every N seconds. Default: 1 hour\\
{\tt -{}-verbose, -v}         & Enable more output.\\
{\tt -{}-clear}               & Clear the checksum of the file if it's wrong.\\
{\tt -{}-recoveryscript FILE} & If a checksum error was found, run this script. The character \$ represents the filename as an argument, so e.g. "fixchecksum \$" would run "fixchecksum" with the bad files as the argument. Default: None\\

\hline
\end{tabular}
\caption{Command line arguments}
\label{tab:cmdlineargs}
\end{table}

\subsection{checksum-monitor}
The checksum-monitor main settings file is "settings.py" in the root directory of the checksum-monitor project. The important configuration variables are
\begin{description}
\item[DEBUG] enables or disables debug mode. It may be most reasonable to set this to False for security reasons. In addition, with debug mode enabled, a history of all queries is stored, so that the application in effect leaks memory. This is problematic because it will lead to a crash after some time.
\item[DATABASES] has the database information: type of database (e.g. SQLite or MySQL), database name and path, username/password and so on. As default, it is set up with MySQL on localhost with username root without a password.
\item[MONITOR\_ROOT] is the root directory of the checksum-monitor project. This must be set.
\item[log\_path] is the (log path, extension) tuple for the checksumd error logs.
\item[fixed\_checksums\_path] is the (path, extension) tuple of the logs containing resolved checksum errors.
\item[lost\_files\_path] is the (path, extension) tuple of the logs containing lost files.
\item[stager\_hosts] is a list of the stager hosts used for checking if a file is lost.
\end{description}

After the database is set up and the settings are updated, the tables can be created automatically:

{\tt <MONITOR\_ROOT>/manage.py syncdb} 

All the required tables will then be set up with the proper keys and datatypes. The web server can then be set up to listen on port 80 using

{\tt <MONITOR\_ROOT>/manage.py runserver 0.0.0.0:80} 

However, the recommended way is using a third-party web server like for instance apache, as the built-in web server is not intended for production use.

\subsubsection{Refreshing the database}
When new log entries are added, the database must be refreshed (logs must be parsed again) for the entries to show up in the monitor. There are two ways:
\begin{enumerate}
\item Using the web interface's "Refresh" link.
\item Calling the refresh command directly through manage.py, e.g. through a cron job.
\end{enumerate}

The second method may be the preferred one because it ensures that many people don't call the refresh command at once. The syntax is

{\tt <MONITOR\_ROOT>/manage.py refresh <category>} 

where <category> can be "checksumd", "fixed" or "lostfiles". The log directory for the specified category will then be scanned and new entries will be added.
