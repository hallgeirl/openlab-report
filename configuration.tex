\section{Configuration}

\subsection{checksumd}
All of checksumd's configuration is done by command line arguments. checksumd can be run in two modes: as a daemon, or as a regular command line tool. The available command line arguments are listed in table \ref{tab:cmdlineargs}.

\begin{table}[ht]
\begin{tabular}{lp{10cm}}
\hline
{\bf Argument}              & {\bf Description}\\
\hline
{\tt -{}-daemon}              & Run checksumd as a daemon.\\
{\tt -{}-loopcount N}         & Run for N loops. Default: -1 (infinite) in daemon mode and 1 otherwise.\\
{\tt -{}-nodlf}               & Disable output to syslog.\\
{\tt -{}-checkns}             & Also check the content checksum against the one stored in the name server.\\
{\tt -{}-throttlesleep N}     & Sleep for N seconds if all disks are found to be busy (or done), before doing another busy check. Default: 10 sec.\\
{\tt -{}-maxruntime N}        & Terminate the program after N seconds. Default: infinite\\
{\tt -{}-maxwait N}           & If disks are busy and has M open files, force a check after $N\cdot 2^{M-1}$ seconds. Default: -1 (infinity)\\
{\tt -{}-minrepeattime N}     & If checking all files takes T < N seconds, sleep for N-T seconds before checking them again. Default: 24 hours\\
{\tt -{}-setchecksumifnone}   & Enable in order to set the checksum of files that does not have one.\\
{\tt -{}-modifyignore N}      & Ignore files that have been modified less than N seconds ago. Default: 24 hours\\
{\tt -{}-errorfile DIR}       & Save error logs under the directory DIR. Filenames are given as {\tt checksum\_errors\_<hostname>.log}.\\
{\tt -{}-nthreads N}          & Use N worker threads for integrity checking. This does not include the main thread. Default: 2\\
{\tt -{}-refreshrate N}       & Refresh the file queues every N seconds. Default: 1 hour\\
{\tt -{}-verbose, -v}         & Enable more output.\\
{\tt -{}-clear}               & Clear the checksum of the file if it's wrong.\\
{\tt -{}-recoveryscript FILE} & If a checksum error was found, run this script. The character \$ represents the filename as an argument, so e.g. "fixchecksum \$" would run "fixchecksum" with the bad files as the argument. Default: None\\

\hline
\end{tabular}
\caption{Command line arguments}
\label{tab:cmdlineargs}
\end{table}

A default configuration file, /etc/sysconfig/checksumd is shipped with the RPM containing the daemon. This is a basic shell script that puts the command line arguments into an environment variable \$CHECKSUMD\_OPTIONS that is used by checksumd's runscript (typically /etc/init.d/checksumd). E.g., setting \$CHECKSUMD\_OPTIONS to -{}-maxwait 90, causes the runscript to run "checksumd -{}-maxwait 90 -{}-daemon" (daemon mode is implied by the init.d script).

\subsection{checksum-monitor}
The checksum monitor is more or less a standard Django application, and thus its configuration can be found in <django\_project\_root>/settings.py. In addition to the standard settings like database paths and passwords, the following values must be configured properly:
\begin{itemize}
\item MONITOR\_ROOT: Points to the Django project root directory, e.g. /www/checksum-monitor.
\item APP\_ROOT: Relative path from MONITOR\_ROOT to where the application itself (views.py, models.py, ...) is contained, e.g. /monitor/.
\item Log paths; tuples of (PATH, EXT) where PATH points to the log directory described below, and EXT is the file extension of the logs in that file (e.g. ".log"):
\begin{itemize}
\item log\_path: PATH points to the log directory where error logs from checksumd is stored.
\item fixed\_checksums\_path: PATH points to the log directory where resolved checksum errors are stored.
\item lost\_files\_path: PATH points to the log directory where lost file logs are stored.
\end{itemize}
\end{itemize}

After the configuration is set up properly, there are two ways of refreshing the database of events: Running "python manage.py refresh" from the project root directory, or using the Refresh button in the web interface. All files in the specified directories will then be read and parsed, and rows will be inserted into the database. 
