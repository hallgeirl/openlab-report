\abstract
For storing physics data from the experiments at CERN, a hierarchial system of tapes and disks are used where tapes are the main media of storage, while disks are used as a cache. Making sure that the files on the disks are not corrupted are important in order to prevent bad files to be written back to tape. Typically this is done by computing a checksum of the files and compare it with a stored checksum. In my project I implement a checksum daemon based on previous work that aims to continuously check files for corruption and provide logs for external applications, with minimal impact on performance of the disk servers. I also implemented monitoring through LeMon (LHC Era MONitoring) that gives general statistics, e.g. files checked, etc., and I also developed a monitoring application in Django for more detailed statistics on a per-error basis.
