\section{Implementation}
\subsection{checksumd}
Disk queue
File queue
Threads

\subsubsection{Logging}
checksumd may output log entries to multiple logging destinations, like the syslog, standard output (if running in non-daemon mode) and a flat file with file errors. Because of this, a unified log manager was developed that handles output to each log. One class, LogManager, are responsible for taking in a log entry and storing it in each log.  

\subsection{checksum-monitor}
Views
Models
Templates
Static files

\subsection{Lemon integration}
\label{sec:lemon_integration}
Monitoring of checksumd summaries are done using Lemon. More specifically, the metric ParseExtract in the sensor ParseLog was used. ParseExtract parses a log file, attempts to find a match for the regular expression given to the metric, and extracts specified values from each line. 

In the case of checksumd, the following strategy is used: Every hour, the sensor will get the matching entries from the last hour, which should be either 0 or 1 entry as long as the reporting rate from the daemon is larger than one hour. It extracts the key numbers (\# files scanned, \# bad checksums found, ...) and stores them in the same order that they are found. This is not a problem because the key values will always be output in the same order. This data are then sent to the Lemon server by the agent.
